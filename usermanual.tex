\documentclass[12pt]{article}
\usepackage{fullpage,amsmath,amsfonts,mathpazo,microtype,nicefrac}

% Set-up for hypertext references
\usepackage{hyperref,color,textcomp}
\definecolor{webgreen}{rgb}{0,.35,0}
\definecolor{webbrown}{rgb}{.6,0,0}
\definecolor{RoyalBlue}{rgb}{0,0,0.9}
\hypersetup{
   colorlinks=true, linktocpage=true, pdfstartpage=3, pdfstartview=FitV,
   breaklinks=true, pdfpagemode=UseNone, pageanchor=true, pdfpagemode=UseOutlines,
   plainpages=false, bookmarksnumbered, bookmarksopen=true, bookmarksopenlevel=1,
   hypertexnames=true, pdfhighlight=/O,
   urlcolor=webbrown, linkcolor=RoyalBlue, citecolor=webgreen,
   pdfauthor={Jane Huang, Kimia Mavon, Weidong Xu, Jeffrey Zhao},
   pdfsubject={chemkin User Manual},
   pdfkeywords={},
   pdfcreator={pdfLaTeX},
   pdfproducer={LaTeX with hyperref}
}
\hypersetup{pdftitle={User Manual}}

% Macro definitions
\newcommand{\N}{\mathbb{N}}
\newcommand{\Z}{\mathbb{Z}}
\newcommand{\Q}{\mathbb{Q}}
\newcommand{\R}{\mathbb{R}}
\newcommand{\p}{\partial}
\renewcommand{\vec}[1]{\mathbf{#1}}
\newcommand{\vx}{\vec{x}}
\newcommand{\vp}{\vec{p}}
\newcommand{\Trans}{\mathsf{T}}

\begin{document}
\title{The \textsc{chemkin} User Manual}
\author{Jane Huang, Kimia Mavon, Weidong Xu, Jeffrey Zhao}
\date{}
\maketitle
\section{Introduction}


\textsc{chemkin} is a Python library that computes the reaction rates of all species participating in a system of elementary, irreversible reactions. 

\subsection{Key chemical concepts and terminology}

A system consisting of $M$ elementary reactions involving $N$ species has the general form 
\begin{align}
  \sum_{i=1}^{N}{\nu_{ij}^{\prime}\mathcal{S}_{i}} \longrightarrow 
  \sum_{i=1}^{N}{\nu_{ij}^{\prime\prime}\mathcal{S}_{i}}, \qquad j = 1, \ldots, M.
\end{align}

$S_i$ is the $i$th specie in the system, $\nu_{ij}^{\prime}$ is its stoichiometric coefficient (dimensionless) on the reactants side of the $j$th reaction, and $\nu_{ij}^{\prime\prime}$ is its stoichiometric coefficient (dimensionless) on the product side for the $j$th reaction. 

Each specie is characterized by a concentration $x_i$, in units of [mol/vol]. 
The \textbf{reaction rate} of each specie is the time rate of change of its concentration, $\frac{dx_i}{dt}$. The reaction rate is usually represented by the symbol $f_i$, such that 
\begin{align}
  f_{i} = \sum_{j=1}^{M}{(\nu_{ij}^{\prime\prime}-\nu_{ij}^\prime)\omega_{j}}= \sum_{j=1}^{M}{\nu_{ij}\omega_{j}}, \qquad i = 1, \ldots, N.
\end{align}

$\omega_j$ is the \textbf{progress rate} of the $j$th reaction, 
\begin{align}
  \omega_{j} = k_{j}\prod_{i=1}^{N}{x_{i}^{\nu_{ij}^{\prime}}}, \qquad j = 1, \ldots, M. 
\end{align}
\subsection{Features}

The package can solve for the reaction rates of a system with an arbitrary number of species and elementary reactions. The reaction rate coefficient $k$ for each reaction is assumed to take one of three possible forms:
\begin{enumerate}
\item $k=$ constant
\item Arrhenius: $k=A\exp(-\frac{E}{RT})$, where $A$ is the pre-factor, $E$ is the activation energy, $R$ is the universal gas constant, and $T$ is the temperature. 
\item Modified Arrhenius: $k=AT^b\exp(-\frac{E}{RT})$, where $A$ is the pre-factor, $E$ is the activation energy, $R$ is the universal gas constant, $T$ is the temperature, and $b$ is the temperature scaling parameter. 
\end{enumerate}

\subsubsection{Input}
\subsubsection{Output}
\subsubsection{Installation} 

Describe where the code can be found and downloaded. Tell the user how to run the test suite. We are not releasing this code as a package yet, but when we do that this section will include instructions how how to install the package.

\section{Basic Usage and Examples}

\subsection{Input format}

Provide a few examples on using your software in some common situations. You may want to show how the code works with a small set of reactions.



\end{document}
